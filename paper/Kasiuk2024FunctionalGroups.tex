\documentclass{article}
\usepackage{arxiv}

\usepackage[utf8]{inputenc}
\usepackage[english, russian]{babel}
\usepackage[T2A, T1]{fontenc}
\usepackage{url}
\usepackage{booktabs}
\usepackage{amsfonts}
\usepackage{nicefrac}
\usepackage{microtype}
\usepackage{lipsum}
\usepackage{graphicx}
\usepackage{natbib}
\usepackage{doi}



\title{Восстановление функциональных групп головного мозга с помощью графовых диффузных моделей}

\author{
	Касюк Вадим \\
	\texttt{kasiuk.va@phystech.edu} \\
	%% examples of more authors
	\And
	Игнатьев Даниил \\
	\texttt{ignatev.da@phystech.edu} \\
	\And
	Панченко Святослав \\
	\texttt{panchenko.sk@phystech.edu}
	%% \AND
	%% Coauthor \\
	%% Affiliation \\
	%% Address \\
	%% \texttt{email} \\
	%% \And
	%% Coauthor \\
	%% Affiliation \\
	%% Address \\
	%% \texttt{email} \\
	%% \And
	%% Coauthor \\
	%% Affiliation \\
	%% Address \\
	%% \texttt{email} \\
}
\date{}

\renewcommand{\shorttitle}{Ускоренные методы нулевого порядка}
\renewcommand{\undertitle}{}
%%% Add PDF metadata to help others organize their library
%%% Once the PDF is generated, you can check the metadata with
%%% $ pdfinfo template.pdf


\begin{document}
\maketitle

\begin{abstract}
Рассматривается задача классификации активности человека по сигналам электроэнцефалограммы его головного мозга. Классические подходы, основанные на свёрточных сетях, не учитывают пространственную структуру сигнала, регистрируемого датчиками, и неизбежно теряют информацию. Предлагается осуществить классификацию сигналов мозга на основе графовых представлений его функциональных групп, а в качестве предсказательной модели использовать графовую нейронную диффузию GRAND и BLEND. Полученные результаты показывают, что предлагаемое решение превосходит в качестве существующие аналоги.

\end{abstract}


\keywords{Головной мозг, ЭЭГ, нейроные сети, дуффизионые модели}

\section{Введение}

\bibliographystyle{unsrtnat}
%\bibliography{references}

\end{document}
