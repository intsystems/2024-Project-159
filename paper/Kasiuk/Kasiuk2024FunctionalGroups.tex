\documentclass{article}
\usepackage{arxiv}

\usepackage[utf8]{inputenc}
\usepackage[english, russian]{babel}
\usepackage[T2A, T1]{fontenc}
\usepackage{url}
\usepackage{booktabs}
\usepackage{amsfonts}
\usepackage{nicefrac}
\usepackage{microtype}
\usepackage{lipsum}
\usepackage{graphicx}
\usepackage{natbib}
\usepackage{doi}



\title{Восстановление функциональных групп головного мозга с помощью графовых диффузных моделей}

\author{
	Касюк Вадим \\
	\texttt{kasiuk.va@phystech.edu} \\
	%% examples of more authors
	\And
	Игнатьев Даниил \\
	\texttt{ignatev.da@phystech.edu} \\
	\And
	Панченко Святослав \\
	\texttt{panchenko.sk@phystech.edu}
	%% \AND
	%% Coauthor \\
	%% Affiliation \\
	%% Address \\
	%% \texttt{email} \\
	%% \And
	%% Coauthor \\
	%% Affiliation \\
	%% Address \\
	%% \texttt{email} \\
	%% \And
	%% Coauthor \\
	%% Affiliation \\
	%% Address \\
	%% \texttt{email} \\
}
\date{}

\renewcommand{\shorttitle}{Ускоренные методы нулевого порядка}
\renewcommand{\undertitle}{}
%%% Add PDF metadata to help others organize their library
%%% Once the PDF is generated, you can check the metadata with
%%% $ pdfinfo template.pdf


\begin{document}
\maketitle

\begin{abstract}
Рассматривается задача классификации активности человека по сигналам электроэнцефалограммы его головного мозга. Классические подходы, основанные на свёрточных сетях, не учитывают пространственную структуру сигнала, регистрируемого датчиками, и неизбежно теряют информацию. Предлагается осуществить классификацию сигналов мозга на основе графовых представлений его функциональных групп, а в качестве предсказательной модели использовать графовую нейронную диффузию GRAND и BLEND. Полученные результаты показывают, что предлагаемое решение превосходит в качестве существующие аналоги.

\end{abstract}


\keywords{Головной мозг, ЭЭГ, нейроные сети, дуффизионые модели}

\section{Введение}

Целью исследования является построение функциональных групп головного мозга с помощью современных графовых диффузионных моделей. А также на основании анализа полученных результатов предполагается доказать интерпретируемость ЭЭГ исследований для разных людей, что позволит экспертам в области нейробиологии разрабатывать более эффективные методы лечения патологий головного мозга.

Обьектом исследования являются простраственные временные ряды полученные с помощью ЭЭГ головного мозга группы испытуемых. Устройство для считывания сигнала представляет собой
набор датчиков – электродов, расположенных на поверхности кожи головы по одной
из общепринятых систем размещения. На выходе получается меняющая во времени с определенной частотой дискретизации матрица с значениями интенсивности датчиков в ее ячейках.

Проблематика данной задачи заключается в том, что предыдущие исследования не учитывают пространственную структуру сигнала, что не позволяет добиться высокой точности. Более того, даже те модели, которые пытаются учесть эту особенность, не учитывают динамическую зависимость этих групп, что так же ограничивает точность результатов.

Автор статьи ставит перед собой задачи : 1) Повторить эскперименты Святослава Панченко и Михаила Бронштейна(тут ссылочки на статьи)
2) Решить проблему гладкости графа во времени 
3) Построить метрическое пространство для графов для использования данных в численном эксперименте в диффузионной модели
4) Провести эксперимент используя графовую нейронную диффузию GRAND и
BLEND.
5) Провести сранительный анализ с предыдущими решениями

\section{Постановка задачи}
Дана выборка $\mathcal{D} = (\underline{\mathbf{X}}, \mathbf{Z}, \mathbf{y})$ активности головного мозга, где
\[ \underline{\mathbf{X}} =[︀X_m]︀_{m=1}^M \textrm{ -- набор сигналов,}\]
\[\mathbf{X}_m=[\mathbf{x}_t]_{t\in T}  \textrm{ -- сигнал, полученный в} m\textrm{-ом испытании,}\]
\[\mathbf{x}_t \in \mathbb{R}^E \textrm{ -- наблюдения сигнала в момент времени }t, \]
\[\mathbf{Z} = [\mathbf{z}_k]_{k=1}^E
z_k \in \mathbf{R}^3 \textrm{ -- координаты электродов},\]
\[\mathbf{y} = [y_m]_{m=1}^M \textrm{ -- целевая переменная,}\]
\[y_m \in {1,... C} \textrm{ -- метка класса},\]
\[T = \{t_n\}_{n=1}^N \textrm{ -- набор временных отсчетов},\]
\[E = 62 \textrm{ -- число электродов,}\]
\[N \textrm{ -- число наблюдений в одном отрезке сигнала.}\]
Для устойчивости решения на матрицу связности накладывается штраф за плотность:
\[\underline{\mathbf{A}}^*_{\underline{\mathbf{X}},\mathbf{Z}} = \arg \min_{\underline{\mathbf{A}}_{\underline{\mathbf{X}},\mathbf{Z}}} \bigg| \| \underline{\mathbf{A}}_{\underline{\mathbf{X}},\mathbf{Z}}\|_1 -p\bigg|\]
где $p$ соответствует степени разреженности связей и подбирается из априорных предположений.
Для решения задачи декодирования рассматривается модель из класса графовых рекуррентных нейронный сетей:
\[h_{\mathbf{\theta}} : (\underline{\mathbf{X}}, \underline{\mathbf{A}}^*_{\underline{\mathbf{X}},\mathbf{Z}}) \rightarrow y. \]
В качестве функции ошибки выбрана кросс–энтропия:
\[\mathcal{L} = - \frac{1}{M} \sum\limits_{m=1}^M\bigg[\sum\limits_{c=1}^C\mathbbm{1}(y_m=c)\log(p_m^c)\bigg],\] 
где
\[ p_m^c = h_{\mathbf{\theta}}\bigg(\mathbf{X}_m, \underline{\mathbf{A}}^*_{\underline{\mathbf{X}},\mathbf{Z}}(m)\bigg) \textrm{-- вероятность класса} c\]
\[\textrm{для } X_m \textrm{ с матрицей } \underline{\mathbf{A}}^*_{\underline{\mathbf{X}},\mathbf{Z}}(m)\]
Задача поиска оптимальных параметров имеет следующий вид:
\[\hat\mathbf{\theta} = \arg\max_\mathbf{\theta} \mathcal{L}(\mathbf{\theta}, \mathbf{X}, \underline{\mathbf{A}}^*_{\underline{\mathbf{X}},\mathbf{Z}})\] 
%\bibliography{references}



\bibliographystyle{unsrtnat}
%\bibliography{references}

\end{document}
