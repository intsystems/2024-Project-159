\documentclass{article}
\usepackage{arxiv}

\usepackage[utf8]{inputenc}
\usepackage[english, russian]{babel}
\usepackage[T2A, T1]{fontenc}
\usepackage{url}
\usepackage{booktabs}
\usepackage{amsfonts}
\usepackage{nicefrac}
\usepackage{microtype}
\usepackage{lipsum}
\usepackage{graphicx}
\usepackage[square,numbers]{natbib}

\usepackage{hyperref} 
\hypersetup{backref=true,       
    pagebackref=true,               
    hyperindex=true,                
    colorlinks=true,                
    breaklinks=true,                
    urlcolor= black,                
    linkcolor= green,                
    bookmarks=true,                 
    bookmarksopen=false,
    filecolor=black,
    citecolor=black,
    linkbordercolor=blue
}
% \AtBeginDocument{\hypersetup{pdfborder={0 0 1}}}

\usepackage{doi}
\usepackage{bbm}
\usepackage{bm}


\title{Восстановление функциональных групп головного мозга с помощью графовых диффузных моделей}

\author{
    Игнатьев Даниил \\
	\texttt{ignatev.da@phystech.edu} \\
	%% examples of more authors
	\And
	Касюк Вадим \\
	\texttt{kasiuk.va@phystech.edu} \\
	\And
	Панченко Святослав \\
	\texttt{panchenko.sk@phystech.edu}
	%% \AND
	%% Coauthor \\
	%% Affiliation \\
	%% Address \\
	%% \texttt{email} \\
	%% \And
	%% Coauthor \\
	%% Affiliation \\
	%% Address \\
	%% \texttt{email} \\
	%% \And
	%% Coauthor \\
	%% Affiliation \\
	%% Address \\
	%% \texttt{email} \\
}
\date{}

\renewcommand{\shorttitle}{Ускоренные методы нулевого порядка}
\renewcommand{\undertitle}{}
%%% Add PDF metadata to help others organize their library
%%% Once the PDF is generated, you can check the metadata with
%%% $ pdfinfo template.pdf


\begin{document}
\maketitle

\begin{abstract}
Рассматривается задача классификации временных рядов, представляющих собой электроэнцефалограмму головного мозга человека. Расположение датчиков, регистрирующих сигнал, на поверхности полусферы не оставляет возможности учесть пространственную структуру сигнала с помощью двумерных сверточных фильтров. Вместо свёрточных сетей предлагается использовать подход на основе графового представления функциональных групп мозга, получаемого на основе имеющегося сигнала. В качестве модели предлагается использовать графовую нейронную диффузию. Результаты экспериментов показывают, что предложенное решение превосходит классические подходы.

\end{abstract}


\keywords{Головной мозг, ЭЭГ, нейронные сети, диффузионные модели}

\section{Введение}

Данная работа посвящена исследованию использования пространственной структуры сигналов активности головного мозга для решения задач декодирования и направлена на сравнение различных методов декодирования \cite{gcnsnet2020, Hou_2020, LI2023833}. Одним из распространенных методов для получения нейронной активности мозга является электроэнцефалограмма (EEG или ЭЭГ), измеряющая колебания
напряжения, спровоцированного электрическим током в нейронах мозга. В данном исследовании в качестве задачи декодирования рассматривается классификация сигналов ЭЭГ для анализа эмоциональных состояний человека. Устройство для считывания сигнала представляет собой
набор датчиков – электродов, расположенных на поверхности кожи головы по одной
из общепринятых систем размещения. Мозг представляет собой динамическую систему, в которой информация постоянно обрабатывается и передается в другие взаимосвязанные регионы. Группы
активности составляют сложную сеть с иерархической пространственной и функциональной организацией, которую необходимо изучить. 
\par 
Сверточные нейронные
сети (CNN) часто используются в качестве модели извлечения пространственной информации многомерных временных рядов мозговой активности. \cite{Hou_2020, LI2023833} CNN по построению применимы к
данным, структурированным в виде упорядоченной регулярной сетки, как изображения, где пиксели равноудалены от своих соседей. Предположение отсутствия сложной
функциональной нейронной связности электродов приводит к ограниченной производительности и худшей интерпретации функциональных структур. Применение CNN
к данным ЭЭГ с использованием 2D сверток на каждом ЭЭГ испытании в виде псевдо–картинки $\mathbb{R}^{E\times T}$, где $E$ — число
электродов, $T$ — число отсчетов времени с расположением электродов по одной оси матрицы игнорирует их размещение на сферической поверхности головы. 
\par Для эффективного учета пространственной структуры ЭЭГ сигнала предлагается использовать графовое представление, где вершины это электроды со значениями
сигнала в конкретный момент времени, а ребра задаются матрицей смежности по известному алгоритму. Построенная графовая структура сигнала ЭЭГ позволяет применить графовые
нейронные сети (GNN) \cite{gcnsnet2020, chamberlain2021blend, varenik2022} для обнаружения и моделирования внутренней связи
различных участков головного мозга. В данной работе предлагается рассмотреть модели графовой диффузии. Выдвигается гипотеза о том, что они покажут результаты для классификации активности головного мозга лучше, чем модели, не использующие графовое представление сигнала, и более простые графовые сверточные сети (GCN).
\bibliographystyle{unsrtnat}

\section{Постановка задачи}
Дана выборка $\mathcal{D} = (\underline{\mathbf{X}}, \mathbf{Z}, \mathbf{y})$ активности головного мозга, где
\[ \underline{\mathbf{X}} =[︀\mathbf{X}_m]︀_{m=1}^M \textrm{ -- набор сигналов,}\]
\[\mathbf{X}_m=[\mathbf{x}_t]_{t\in T}  \textrm{ -- сигнал, полученный в } m\textrm{-ом испытании,}\]
\[\mathbf{x}_t \in \mathbb{R}^E \textrm{ -- наблюдения сигнала в момент времени }t, \]
\[\mathbf{Z} = [\mathbf{z}_k]_{k=1}^E
z_k \in \mathbf{R}^3 \textrm{ -- координаты электродов},\]
\[\mathbf{y} = [y_m]_{m=1}^M \textrm{ -- целевая переменная,}\]
\[y_m \in {1,... C} \textrm{ -- метка класса},\]
\[T = \{t_n\}_{n=1}^N \textrm{ -- набор временных отсчетов},\]
\[E = 62 \textrm{ -- число электродов,}\]
\[N \textrm{ -- число наблюдений в одном отрезке сигнала.}\]
По выборке $\mathcal{D}$ строится неориентированный динамический граф:
$\mathcal{G}(m, t) = \Big(\mathcal{V}(m, t), \mathcal{E}(m, t), \vect \mathbf{A}_{\underline{\vect \mathbf{X}},\vect \mathbf{Z}}(m, t)\Big),$
где
$$
\mathcal{V}(m, t) \textrm{ -- множество электродов,}
$$
$$
\mathcal{E}(m, t) \textrm{ -- множество ребер,}
$$
$$\vect \mathbf{A}_{\underline{\vect \mathbf{X}},\vect \mathbf{Z}}(m, t) \textrm{ -- матрица смежности графа, определяющая веса ребер },$$
$$
\underline{\mathbf{A}}_{\underline{\vect \mathbf{X}},\vect \mathbf{Z}} = [\mathbf{A}_{\underline{\vect \mathbf{X}},\vect \mathbf{Z}}]_{m=1}^M \textrm{ -- набор матриц смежности.}
$$
Для решения задачи декодирования рассматривается модель из класса графовых рекуррентных нейронный сетей, параметризуемого множеством $\Theta$:
\[h_{\bm{\theta}} : (\underline{\mathbf{X}}, \underline{\mathbf{A}}_{\underline{\mathbf{X}},\mathbf{Z}}) \rightarrow y, \bm{\theta}\in\Theta\]

В качестве функции ошибки выбрана кросс–энтропия:
\[\mathcal{L} = - \frac{1}{M} \sum\limits_{m=1}^M\bigg[\sum\limits_{c=1}^C\mathbbm{1}(y_m=c)\log(p_m^c)\bigg],\] 
где
\[ p_m^c = h_{\mathbf{\theta}}\bigg(\mathbf{X}_m, \underline{\mathbf{A}}_{\underline{\mathbf{X}},\mathbf{Z}}(m)\bigg) \textrm{-- вероятность класса } c\]
\[\textrm{для } \mathbf{X}_m \textrm{ с матрицей } \underline{\mathbf{A}}_{\underline{\mathbf{X}},\mathbf{Z}}(m).\]
Задача поиска оптимальных параметров имеет следующий вид:
\[\hat\mathbf{\theta} = \arg\max_{\bm{\theta}\in\Theta} \mathcal{L}(\bm{\theta}, \mathbf{X}, \underline{\mathbf{A}}_{\underline{\mathbf{X}},\mathbf{Z}})\]

\section{Предложенное решение}
Предлагается следующий пайплайн GNN+LSTM:
\begin{enumerate}
    \item По исходным данным $\mathcal{D}$ строится неориентированный динамический граф $\mathcal{G}(m, t)$.
    \item Граф $\mathcal{G}(m, t)$ подается на вход GNN. На выходе получаем многомерный временной ряд представлений исходных данных.
    \item На выходах GNN запускаем LSTM, по его скрытому состоянию после обработки входа производим классификацию
\end{enumerate}
Для построения матрицы смежности графа будем использовать методы, исследованные в \cite{varenik2022}. Лучшие результаты были получены методом на основе корреляции Пирсона двух сигналов, поэтому будем использовать именно его. Также важным гиперпараметром является размер окна входных данных, по которому будем строить граф.
\par
В качестве GNN предлагается использовать графовую диффузию BLEND. В основе алгоритма лежит связь с физическим уравнением диффузии:
$$
\frac{\partial x(u,t)}{\partial t} = \text{div}[g(u, t, x(u,t))\nabla_u x],
$$
где $x(u,t)$ -- величина потока (тепла) в среде в точке пространства $u$ в момент времени $t$. \\Схожим образом определяется диффузия на графе. Пусть $\mathcal{G} = (\mathcal{V}=\{1,...,n\},\mathcal{E})$ неориентированный граф с множествами вершин и ребер $\mathcal{V}$ и $\mathcal{E}$ соответственно. Обозначим через $\mathbf{z}_i(t) = (\mathbf{u}_i, \mathbf{x}_i)\in\mathbb{R}^d\times\mathbb{R}^{d'}$ объедиененные координаты $i$-ой вершины. Тогда уравнение диффузии на графе имеет вид:
$$ \frac{\partial{\mathbf{z}_i(t)}}{\partial{t}} = div [a(\mathbf{z}(t))\nabla \mathbf{z}_i(t)],
$$
где $a(\mathbf{z}(t))$ -- диффузивность, определяет интенсивность процесса вдоль различных направлений. Найдем решение уравнения, переписав правую часть равенства, используя определения операторов дивергенции и градиента на графе и проведя дискретизацию дифференциального уравнения:
$$
    \frac{z^{(k+1)}_i - z^{(k)}_i}{\tau} = \sum\limits_{j:(i,j)\in\mathcal{E}(U^{(k)})}a(z^{(k)}_i, z^{(k)}_j)(z^{(k)}_j - z^{(k)}_i) 
$$
При $\tau = 1$ уравнение компактно переписывается в виде явной схемы Эйлера:

$$ Z^{(k+1)} = (A^{(k)} - I)Z^{(k)} = Q^{(k)}Z^{(k)}$$
Решение вычисляется последовательным применением схемы несколько раз подряд. Схема BLEND обобщает многие другие подходы, в том числе может воспроизводить GCN \cite{chamberlain2021blend}, при этом использует меньше параметров, чем GCN.

\section{Вычислительный эксперимент}
Нами ставятся две гипотезы:
\begin{enumerate}
    \item Утилизация пространственной информации позволит GNN показать результат лучше, чем GNN при решении задачи классификации.
    \item Модель BLEND будет более робастной, чем GCN за счет меньшего количества параметров при том же качестве.
\end{enumerate}
Целью эксперимента является проверить эти гипотезы. \\
Датасет для экспериментов: EEG Database to examine EEG correlates of genetic
predisposition to alcoholism -- данные ЭЭГ испытуемых из двух групп: страдающих алкоголизмом и не страдающих. \href{https://archive.ics.uci.edu/dataset/121/eeg+database}{Ссылка на датасет}
\\
В ходе эксперимента были обучены модели GCN для нескольких размеров окна.
\newpage
\begin{flushleft}
Кривые обучения и значения метрик на тесте для GCN.
\end{flushleft}
\begin{figure}[!h]
\begin{minipage}{.475\textwidth} 
    \includegraphics[width=\textwidth]{t5(1).png}
 	\caption{Кривые обучения}
\end{minipage}
\begin{minipage}{.475\textwidth}
    \includegraphics[width=\textwidth]{final_res(1).png}
 	\caption{Значения метрик}
\end{minipage}
\end{figure}

\section{Заключение}
В дальнейшем планируется довести эксперимент до конца и проверить поставленные гипотезы.

\bibliography{references}
\end{document}

