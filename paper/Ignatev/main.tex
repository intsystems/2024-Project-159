\documentclass{article}
\usepackage{arxiv}

\usepackage[utf8]{inputenc}
\usepackage[english, russian]{babel}
\usepackage[T2A, T1]{fontenc}
\usepackage{url}
\usepackage{booktabs}
\usepackage{amsfonts}
\usepackage{nicefrac}
\usepackage{microtype}
\usepackage{lipsum}
\usepackage{graphicx}
\usepackage{natbib}
\usepackage{doi}
\usepackage{bbm}



\title{Восстановление функциональных групп головного мозга с помощью графовых диффузных моделей}

\author{
    Игнатьев Даниил \\
	\texttt{ignatev.da@phystech.edu} \\
	%% examples of more authors
	\And
	Касюк Вадим \\
	\texttt{kasiuk.va@phystech.edu} \\
	\And
	Панченко Святослав \\
	\texttt{panchenko.sk@phystech.edu}
	%% \AND
	%% Coauthor \\
	%% Affiliation \\
	%% Address \\
	%% \texttt{email} \\
	%% \And
	%% Coauthor \\
	%% Affiliation \\
	%% Address \\
	%% \texttt{email} \\
	%% \And
	%% Coauthor \\
	%% Affiliation \\
	%% Address \\
	%% \texttt{email} \\
}
\date{}

\renewcommand{\shorttitle}{Ускоренные методы нулевого порядка}
\renewcommand{\undertitle}{}
%%% Add PDF metadata to help others organize their library
%%% Once the PDF is generated, you can check the metadata with
%%% $ pdfinfo template.pdf


\begin{document}
\maketitle

\begin{abstract}
Рассматривается задача классификации временных рядов, представляющих собой электроэнцефалограмму головного мозга человека. Расположение датчиков, регистрирующих сигнал, на поверхности полусферы не оставляет возможности учесть пространственную структуру сигнала с помощью двумерных сверточных фильтров. Вместо свёрточных сетей предлагается использовать подход на основе графового представления функциональных групп мозга, получаемого на основе имеющегося сигнала. В качестве модели предлагается использовать BLEND. Результаты экспериментов показывают, что предложенное решение превосходит классические подходы.

\end{abstract}


\keywords{Головной мозг, ЭЭГ, нейронные сети, диффузионные модели}

\section{Введение}
Данная работа посвящена исследованию использования пространственной структуры сигналов активности головного мозга для решения задач декодирования и направлена на сравнение различных методов декодирования. Одним из самых распространенных методов для получения нейронной активности мозга является электроэнцефалограмма (EEG или ЭЭГ), измеряющая колебания
напряжения, спровоцированного электрическим током в нейронах мозга. В данном исследовании в качестве задачи декодирования рассматривается классификация сигналов ЭЭГ для анализа эмоциональных состояний человека. Устройство для считывания сигнала представляет собой
набор датчиков – электродов, расположенных на поверхности кожи головы по одной
из общепринятых систем размещения. Мозг представляет собой динамическую систему, в которой информация постоянно обрабатывается и передается в другие взаимосвязанные регионы. Группы
активности составляют сложную сеть с иерархической пространственной и функциональной организацией, которую необходимо изучить. Сверточные нейронные
сети (CNN) часто используются в качестве модели извлечения пространственной информации многомерных временных рядов мозговой активности. CNN по построению применимы к
данным, структурированным в виде упорядоченной регулярной сетки, как изображения, где пиксели равноудалены от своих соседей. Предположение отсутствия сложной
функциональной нейронной связности электродов приводит к ограниченной производительности и худшей интерпретации функциональных структур. Применение CNN
к данным ЭЭГ с использованием 2D сверток на каждом ЭЭГ испытании в виде псевдо–картинки $\mathbb{R}^{E\times T}$, где $E$ — число
электродов, $T$ — число отсчетов времени с расположением электродов по одной оси матрицы игнорирует их размещение на сферической поверхности головы. Для эффективного учета пространственной структуры ЭЭГ сигнала предлагается использовать графовое представление, где вершины это электроды со значениями
сигнала в конкретный момент времени, а ребра задаются матрицей смежности по известному алгоритму. Построенная графовая структура сигнала ЭЭГ позволяет применить графовые
нейронные сети (GNN) для обнаружения и моделирования внутренней связи
различных участков головного мозга. В данной работе предлагается рассмотреть модели графовой диффузии. Выдвигается гипотеза о том, что они покажут результаты для классификации активности головного мозга лучше, чем модели, не использующие графовое представление сигнала, и более простые графовые сверточные сети (GCN).
\bibliographystyle{unsrtnat}

\section{Постановка задачи}
Дана выборка $\mathcal{D} = (\underline{\mathbf{X}}, \mathbf{Z}, \mathbf{y})$ активности головного мозга, где
\[ \underline{\mathbf{X}} =[︀X_m]︀_{m=1}^M \textrm{ -- набор сигналов,}\]
\[\mathbf{X}_m=[\mathbf{x}_t]_{t\in T}  \textrm{ -- сигнал, полученный в} m\textrm{-ом испытании,}\]
\[\mathbf{x}_t \in \mathbb{R}^E \textrm{ -- наблюдения сигнала в момент времени }t, \]
\[\mathbf{Z} = [\mathbf{z}_k]_{k=1}^E
z_k \in \mathbf{R}^3 \textrm{ -- координаты электродов},\]
\[\mathbf{y} = [y_m]_{m=1}^M \textrm{ -- целевая переменная,}\]
\[y_m \in {1,... C} \textrm{ -- метка класса},\]
\[T = \{t_n\}_{n=1}^N \textrm{ -- набор временных отсчетов},\]
\[E = 62 \textrm{ -- число электродов,}\]
\[N \textrm{ -- число наблюдений в одном отрезке сигнала.}\]
Для устойчивости решения на матрицу связности накладывается штраф за плотность:
\[\underline{\mathbf{A}}^*_{\underline{\mathbf{X}},\mathbf{Z}} = \arg \min_{\underline{\mathbf{A}}_{\underline{\mathbf{X}},\mathbf{Z}}} \bigg| \| \underline{\mathbf{A}}_{\underline{\mathbf{X}},\mathbf{Z}}\|_1 -p\bigg|\]
где $p$ соответствует степени разреженности связей и подбирается из априорных предположений.
Для решения задачи декодирования рассматривается модель из класса графовых рекуррентных нейронный сетей:
\[h_{\mathbf{\theta}} : (\underline{\mathbf{X}}, \underline{\mathbf{A}}^*_{\underline{\mathbf{X}},\mathbf{Z}}) \rightarrow y. \]
В качестве функции ошибки выбрана кросс–энтропия:
\[\mathcal{L} = - \frac{1}{M} \sum\limits_{m=1}^M\bigg[\sum\limits_{c=1}^C\mathbbm{1}(y_m=c)\log(p_m^c)\bigg],\] 
где
\[ p_m^c = h_{\mathbf{\theta}}\bigg(\mathbf{X}_m, \underline{\mathbf{A}}^*_{\underline{\mathbf{X}},\mathbf{Z}}(m)\bigg) \textrm{-- вероятность класса} c\]
\[\textrm{для } X_m \textrm{ с матрицей } \underline{\mathbf{A}}^*_{\underline{\mathbf{X}},\mathbf{Z}}(m)\]
Задача поиска оптимальных параметров имеет следующий вид:
\[\hat\mathbf{\theta} = \arg\max_\mathbf{\theta} \mathcal{L}(\mathbf{\theta}, \mathbf{X}, \underline{\mathbf{A}}^*_{\underline{\mathbf{X}},\mathbf{Z}})\] 
%\bibliography{references}


\end{document}

